\documentclass[12pt]{article}
\usepackage[utf8]{inputenc}
\usepackage{graphicx}
\usepackage[absolute,overlay]{textpos} % For positioning the logo
\usepackage{geometry}

\geometry{a4paper, margin=2.5cm}
\begin{document}
	\begin{titlepage}
		% Logo in top-right corner — adjusted position and size
		\begin{textblock*}{3cm}(14.5cm,1cm) % {width}(x, y)
			\includegraphics[width=4cm]{images/IPCA-Logo_v2.png}
		\end{textblock*}
		
		\centering
		\vspace*{1cm}
		{\scshape\Large Instituto Politécnico do Cávado e do Ave \par}
		{\scshape\large Escola Superior de Tecnologia\par}
		\vspace{2cm}
		
		{\huge\bfseries Trabalho Prático de Processamento de Linguagens \par}
		\vspace{2cm}
		
		\begin{flushleft}
			\textbf{Autores:} \\
			Diogo Machado nº26042 \\
			Enrique Rodrigues nº28602 \\
			José Alves nº279967
		\end{flushleft}
		
		\vfill
		
		\begin{flushright}
			\textbf{Docente:} Óscar Ribeiro\\
			\textbf{Data:} Maio de 2025
		\end{flushright}
	\end{titlepage}
	
\begin{abstract}
Este artigo descreve o desenvolvimento de um interpretador para a linguagem CQL (Cassandra Query Language), 
implementado em Python. O interpretador é capaz de processar comandos CQL, realizar análise léxica e sintática, 
e executar operações básicas de manipulação de dados. O projeto demonstra a aplicação prática de conceitos 
de compiladores e processamento de linguagens, incluindo análise léxica, análise sintática, e geração de 
árvores de sintaxe abstrata (AST). São apresentados exemplos práticos de código e visualizações da AST 
para demonstrar o funcionamento do interpretador.
\end{abstract}

\section{Introdução}
A linguagem CQL (Cassandra Query Language) é uma linguagem de consulta semelhante ao SQL, 
especificamente projetada para trabalhar com o Apache Cassandra. Este projeto visa desenvolver 
um interpretador que possa processar e executar comandos CQL básicos, demonstrando os princípios 
fundamentais de processamento de linguagens e compiladores. A implementação foi realizada em Python,
uma linguagem que oferece excelentes ferramentas para processamento de linguagens e análise de texto.

\section{Metodologia}
O interpretador foi desenvolvido em Python, utilizando as seguintes bibliotecas principais:
\begin{itemize}
    \item PLY (Python Lex-Yacc) para análise léxica e sintática
    \item Graphviz para visualização da árvore de sintaxe abstrata
    \item PrettyTable para formatação de resultados em formato tabular
\end{itemize}

A estrutura do projeto foi organizada da seguinte forma:


\section{Implementação}
\subsection{Análise Léxica}
A análise léxica é realizada através de expressões regulares que identificam os diferentes 
tokens da linguagem CQL. O analisador léxico é implementado utilizando a biblioteca PLY, 
que permite definir tokens através de expressões regulares. Segue um exemplo da definição 
de alguns tokens:

\begin{verbatim}
# Tokens
tokens = (
    'CREATE', 'TABLE', 'INSERT', 'INTO', 'SELECT', 'FROM',
    'WHERE', 'DELETE', 'DROP', 'IDENTIFIER', 'NUMBER',
    'STRING', 'COMMA', 'SEMICOLON'
)

# Regras para tokens
t_CREATE = r'CREATE'
t_TABLE = r'TABLE'
t_INSERT = r'INSERT'
t_INTO = r'INTO'
t_SELECT = r'SELECT'
t_FROM = r'FROM'
t_WHERE = r'WHERE'
t_DELETE = r'DELETE'
t_DROP = r'DROP'
t_COMMA = r','
t_SEMICOLON = r';'
\end{verbatim}

\subsection{Análise Sintática}
A análise sintática é implementada usando uma gramática que define a estrutura válida dos 
comandos CQL. A gramática suporta os seguintes tipos de comandos:
\begin{itemize}
    \item CREATE TABLE
    \item INSERT INTO
    \item SELECT
    \item DELETE
    \item DROP TABLE
\end{itemize}

Exemplo da definição de algumas regras gramaticais:

\begin{verbatim}
def p_create_table(p):
    '''create_table : CREATE TABLE IDENTIFIER LPAREN column_defs RPAREN'''
    p[0] = ('CREATE_TABLE', p[3], p[5])

def p_column_defs(p):
    '''column_defs : column_def
                  | column_defs COMMA column_def'''
    if len(p) == 2:
        p[0] = [p[1]]
    else:
        p[0] = p[1] + [p[3]]
\end{verbatim}

\subsection{Árvore de Sintaxe Abstrata (AST)}
A AST é gerada durante a análise sintática e representa a estrutura hierárquica do comando. 
Cada nó da árvore representa uma operação ou elemento do comando, facilitando sua interpretação 
e execução. A visualização da AST é realizada utilizando a biblioteca Graphviz.

Exemplo de um comando CQL e sua representação em AST:

\begin{verbatim}
CREATE TABLE users (
    id INT PRIMARY KEY,
    name TEXT,
    email TEXT
);
\end{verbatim}

A AST gerada para este comando é visualizada como uma árvore onde:
\begin{itemize}
    \item O nó raiz representa o comando CREATE TABLE
    \item Os nós filhos representam o nome da tabela e as definições das colunas
    \item Cada definição de coluna contém o nome, tipo e restrições
\end{itemize}

\section{Resultados}
O interpretador foi testado com diversos comandos CQL, demonstrando sua capacidade de:
\begin{itemize}
    \item Processar comandos sintaticamente corretos
    \item Gerar visualizações da AST
    \item Executar operações básicas de manipulação de dados
    \item Apresentar resultados em formato tabular
\end{itemize}

Exemplo de execução de um comando SELECT:

\begin{verbatim}
SELECT * FROM users WHERE id = 1;
\end{verbatim}

Resultado apresentado em formato tabular:
\begin{verbatim}
+----+--------+------------------+
| id |  name  |      email      |
+----+--------+------------------+
| 1  | João   | joao@email.com  |
+----+--------+------------------+
\end{verbatim}

\section{Conclusão}
O desenvolvimento deste interpretador CQL demonstrou a aplicação prática de conceitos 
fundamentais de compiladores e processamento de linguagens. O projeto pode ser expandido 
para suportar mais funcionalidades da linguagem CQL, tais como:
\begin{itemize}
    \item Suporte a índices secundários
    \item Implementação de transações
    \item Otimização de consultas
    \item Suporte a tipos de dados mais complexos
\end{itemize}

A implementação atual serve como uma base sólida para futuras expansões e melhorias, 
demonstrando a viabilidade de desenvolver um interpretador completo para a linguagem CQL 
utilizando Python e suas bibliotecas de processamento de linguagens.

\end{document}
