\documentclass[12pt]{article}
\usepackage[utf8]{inputenc}
\usepackage{graphicx}
\usepackage[absolute,overlay]{textpos} % For positioning the logo
\usepackage{geometry}

\geometry{a4paper, margin=2.5cm}
\begin{document}
	\begin{titlepage}
		% Logo in top-right corner — adjusted position and size
		\begin{textblock*}{3cm}(14.5cm,1cm) % {width}(x, y)
			\includegraphics[width=4cm]{images/IPCA-Logo_v2.png}
		\end{textblock*}
		
		\centering
		\vspace*{1cm}
		{\scshape\Large Instituto Politécnico do Cávado e do Ave \par}
		{\scshape\large Escola Superior de Tecnologia\par}
		\vspace{2cm}
		
		{\huge\bfseries Trabalho Prático de Processamento de Linguagens \par}
		\vspace{2cm}
		
		\begin{flushleft}
			\textbf{Autores:} \\
			Diogo Machado nº26042 \\
			Enrique Rodrigues nº28602 \\
			José Alves nº279967
		\end{flushleft}
		
		\vfill
		
		\begin{flushright}
			\textbf{Docente:} Óscar Ribeiro\\
			\textbf{Data:} Maio de 2025
		\end{flushright}
	\end{titlepage}	
\section{Introdução}

Com este trabalho prático pretendemos adquirir experiência na conceção e implementação  de analisadores léxicos e sintáticos bem como a definição de ações semânticas que traduzem a linguagem de entrada. Entregamos uma solução que serve de alternativa à linguagem de interrogação de base de dados relacionais, que seja exxecutada sobre ficheiros de texto organizados num formato separado por vírgulas CSV (\textit{Comma Separated Value}).
\section{Metodologia}
O interpretador foi desenvolvido em Python, utilizando as seguintes bibliotecas principais:
\begin{itemize}
    \item \textbf{PLY (Python Lex-Yacc):} para análise léxica e sintática
    \item \textbf{Graphviz:} para visualização gráfica da árvore de sintaxe abstrata
    \item \textbf{PrettyPrinter:} para formatação de resultados em formato tabular
\end{itemize}

A estrutura do projeto foi organizada da seguinte forma:



\section{Implementação}
\subsection{Análise Léxica}
A análise léxica é realizada através de expressões regulares que identificam os diferentes 
tokens da nossa linguagem. O analisador léxico é implementado utilizando a biblioteca PLY, 
que permite definir tokens através de expressões regulares. Segue um exemplo da definição da maioria dos tokens, das palavras reservadas e as regras dos tokens:

\begin{verbatim}
#Lista de tokens
tokens = [
"ID",
"STRING",
"NUMBER",
"ASTERISK",
"COMMA",
"EQUALS",
"NOT_EQUALS",
"LESS_THAN",
"GREATER_THAN",
"LESS_EQUALS",
"GREATER_EQUALS",
"SINGLE_COMMENT",
"MULTI_COMMENT",
"SEMICOLON",
]
 
#Palavras reservadas
reserved = {
"import": "IMPORT",
"table": "TABLE",
"from": "FROM",
"export": "EXPORT",
"as": "AS",
"discard": "DISCARD",
"rename": "RENAME",
"print": "PRINT",
"select": "SELECT",
"where": "WHERE",
"limit": "LIMIT",
"create": "CREATE",
"join": "JOIN",
"using": "USING",
"procedure": "PROCEDURE",
"do": "DO",
"end": "END",
"call": "CALL",
"and": "AND",
}
# Regras para tokens
t_ASTERISK = r"\*"
t_COMMA = r","
t_EQUALS = r"="
t_NOT_EQUALS = r"<>"
t_LESS_THAN = r"<"
t_GREATER_THAN = r">"
t_LESS_EQUALS = r"<="
t_GREATER_EQUALS = r">="
t_SEMICOLON = r";"
\end{verbatim}

\subsection{Análise Sintática}
A análise sintática é implementada usando uma gramática que define a estrutura válida dos comandos CQL. A gramática suporta os seguintes tipos de comandos:
\begin{itemize}
    \item CREATE TABLE
    \item INSERT INTO
    \item SELECT
    \item DELETE
    \item DROP TABLE
\end{itemize}

Exemplo da definição de algumas regras gramaticais:

\begin{verbatim}
def p_create_table(p):
    '''create_table : CREATE TABLE IDENTIFIER LPAREN column_defs RPAREN'''
    p[0] = ('CREATE_TABLE', p[3], p[5])

def p_column_defs(p):
    '''column_defs : column_def
                  | column_defs COMMA column_def'''
    if len(p) == 2:
        p[0] = [p[1]]
    else:
        p[0] = p[1] + [p[3]]
\end{verbatim}

\subsection{Árvore de Sintaxe Abstrata (AST)}
A AST é gerada durante a análise sintática e representa a estrutura hierárquica do comando. 
Cada nó da árvore representa uma operação ou elemento do comando, facilitando sua interpretação 
e execução. Também disponiblizamos uma visualização gráfica da AST, que é realizada utilizando a biblioteca Graphviz.

Exemplo de um comando CQL e sua representação em AST:

\begin{verbatim}
CREATE TABLE users (
    id INT PRIMARY KEY,
    name TEXT,
    email TEXT
);
\end{verbatim}

A AST gerada para este comando é visualizada como uma árvore onde:
\begin{itemize}
    \item O nó raiz representa o comando CREATE TABLE
    \item Os nós filhos representam o nome da tabela e as definições das colunas
    \item Cada definição de coluna contém o nome, tipo e restrições
\end{itemize}

\section{Resultados}
O interpretador foi testado com diversos comandos CQL, demonstrando sua capacidade de:
\begin{itemize}
    \item Processar comandos sintaticamente corretos
    \item Gerar visualizações da AST
    \item Executar operações básicas de manipulação de dados
    \item Apresentar resultados em formato tabular
\end{itemize}

Exemplo de execução de um comando SELECT:

\begin{verbatim}
SELECT * FROM users WHERE id = 1;
\end{verbatim}

Resultado apresentado em formato tabular:
\begin{verbatim}
+----+--------+------------------+
| id |  name  |      email      |
+----+--------+------------------+
| 1  | João   | joao@email.com  |
+----+--------+------------------+
\end{verbatim}

\section{Conclusão}
O desenvolvimento deste interpretador CQL demonstrou a aplicação prática de conceitos 
fundamentais de compiladores e processamento de linguagens. O projeto pode ser expandido 
para suportar mais funcionalidades da linguagem CQL, tais como:
\begin{itemize}
    \item Suporte a índices secundários
    \item Implementação de transações
    \item Otimização de consultas
    \item Suporte a tipos de dados mais complexos
\end{itemize}

A implementação atual serve como uma base sólida para futuras expansões e melhorias, 
demonstrando a viabilidade de desenvolver um interpretador completo para a linguagem CQL 
utilizando Python e suas bibliotecas de processamento de linguagens.

\end{document}
